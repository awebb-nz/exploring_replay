\documentclass[11pt]{article} % For LaTeX2e
\usepackage{rldm,palatino}
\usepackage{graphicx}

\title{Directed exploration with Bayes-adaptive replay}


\author{
Georgy Antonov\\
Department of Computational Neuroscience\\
Max Planck Institute for Biological Cybernetics\\
72076 T\"{u}bingen, Germany\\
\\
Graduate Training Centre of Neuroscience\\
International Max Planck Research School\\
University of T\"{u}bingen\\
72076 T\"{u}bingen, Germany\\
\texttt{georgy.antonov@tuebingen.mpg.de} \\
\And
Peter Dayan \\
Department of Computational Neuroscience\\
Max Planck Institute for Biological Cybernetics\\
72076 T\"{u}bingen, Germany\\
\\
University of T\"{u}bingen\\
72074 T\"{u}bingen, Germany\\
\texttt{dayan@tue.mpg.de} \\
}

% The \author macro works with any number of authors. There are two commands
% used to separate the names and addresses of multiple authors: \And and \AND.
%
% Using \And between authors leaves it to \LaTeX{} to determine where to break
% the lines. Using \AND forces a linebreak at that point. So, if \LaTeX{}
% puts 3 of 4 authors names on the first line, and the last on the second
% line, try using \AND instead of \And before the third author name.

\newcommand{\fix}{\marginpar{FIX}}
\newcommand{\new}{\marginpar{NEW}}

\begin{document}

\maketitle

\begin{abstract}
The \emph{title} should be a maximum of 100 characters. 

The \emph{abstract} should be a maximum of 2000 characters of text,
including spaces (no figure is allowed). You will be asked to copy
this into a text-only box; and it will appear as such in the
conference booklet. Use 11~point type, with a vertical spacing of
12~points.  The word \textbf{Abstract} must be centered, bold, and in
point size 12. Two line spaces precede the abstract.
\end{abstract}

\keywords{
Reinforcement learning, DYNA, planning, exploration, replay
}

\acknowledgements{Put something here.}  


\startmain % to start the main 1-4 pages of the submission.

\section{Plan}

\begin{itemize}
   \item Introduction
   \begin{itemize}
      \item What is replay
      \item DYNA; prioritised sweeping; M\&D -- replay as planning?
      \item Planning and explore-exploit; the lack of exploration in M\&D
      \item Exploration in Bayesian bandits; Gittins indices
      \item MCTS/BAMCP; Prioritised sweeping in belief trees, Bayes-adaptive replay
   \end{itemize}
   \item Results
   \begin{itemize}
      \item Prioritised sweeping in Bayesian bandits
      \begin{itemize}
         \item Expected value of a backup -- probabilistic need
         \item Tree policy and empirical convergence to optimal values
      \end{itemize}
      \item Bayes-adaptive optimised replay
      \begin{itemize}
         \item Expected value of a backup -- joint belief and state dynamics (information states)
         \item We should say something about forgetting here
         \item (Estimated) information gain is subsumed by M\&D's Gain
         \item How information-augmented Gain interacts with Need (example replay in a maze) 
      \end{itemize}
   \end{itemize}
   \item Discussion
\end{itemize}

\end{document}
