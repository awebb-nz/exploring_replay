\documentclass{article}

\usepackage[margin=0.8in]{geometry}
\usepackage{amsmath, amsfonts}
\usepackage{booktabs}
\usepackage{hyperref}
\usepackage{xcolor}
\usepackage{tikz}
\usepackage{graphicx}

\tikzset{block/.style ={rectangle,draw}
}
\usetikzlibrary{positioning, shapes.geometric}
\usetikzlibrary{trees}

\setlength{\parindent}{0pt}

\begin{document}

\section*{More trees}

The EVB for each single-action update (i.e., taking action $a^*$ at belief state $b$) is calculated as:

\begin{equation}
    \text{EVB}(ba^*) = p(b_{\rho}\rightarrow b \mid \pi)\times \big(\mathbb{E}_{b'\sim p(b'\mid b, a^*)}\big[v(ba^*)\big]-v(b)\big)
    \label{eqn:evb}
\end{equation}

Where $p(b_{\rho} \rightarrow b \mid \pi)$ is the Need term -- i.e., the probability of reaching belief $b$ from 
the root of the tree, $b_{\rho}$, when following policy $\pi$. The policy $\pi$ here is i) softmax of the MF 
$Q$-values at the root of the tree, $b_{\rho}$; and ii) softmax of the immediate reward at every other node 
within the tree.
\bigbreak
In the trees, each action has a blue number written above it -- these are the $Q$-values. Each orange square is a belief state, 
and the opacity of each belief state is proportional to the Need term from equation \ref{eqn:evb}. When belief states 
appear in pairs, the top belief state is always the one which results from obtaining a reward from the corresponding 
action, whilst the bottom belief state corresponds to zero reward.
\bigbreak
Updates in the tree are always highlighted by red arrows.

\newpage

\section*{Example 1}
Belief at the root in this example is:

\begin{center}
    \begin{tabular}{c c}
        $\alpha_0 = 5$ & $\beta_0 = 2$ \\
        $\alpha_1 = 1$ & $\beta_1 = 1$
    \end{tabular}
\end{center}

\newpage

\foreach \n in {0, ..., 16}{\input{../data/Tree/seq/0/tex_tree_\n.tex}}

\newpage

\section*{Example 2}
Belief at the root in this example is:

\begin{center}
    \begin{tabular}{c c}
        $\alpha_0 = 5$ & $\beta_0 = 2$ \\
        $\alpha_1 = 2$ & $\beta_1 = 1$
    \end{tabular}
\end{center}

\newpage

\foreach \n in {0, ..., 10}{\input{../data/Tree/seq/1/tex_tree_\n.tex}}

\newpage

\section*{Example 3}
Belief at the root in this example is:

\begin{center}
    \begin{tabular}{c c}
        $\alpha_0 = 10$ & $\beta_0 = 4$ \\
        $\alpha_1 = 2$ & $\beta_1 = 1$
    \end{tabular}
\end{center}

\newpage

\foreach \n in {0, ..., 10}{\input{../data/Tree/seq/1/tex_tree_\n.tex}}

\end{document}
